% Cómo usarlo
% Se debe colocar los archivos de la portada dentro de una carpeta portada
% desde donde será llamado.
%
% Se deben declarar los siguientes comandos, ajustando los valores por
% los que guste:
%
% \newcommand{\HBar}{\rule{\linewidth}{0.5mm}}
% \newcommand{\logoURL}{logoURL.png}
% \newcommand{\nAutor}{Nombre del autor}
% \newcommand{\nCurso}{Curso}
% \newcommand{\cTituloA}{Titulo 1}
% \newcommand{\cTituloB}{Titulo 2}
% \newcommand{\cFecha}{Agosto de 2,014.}

\pagenumbering{gobble}
\begin{titlepage}

\vspace*{\fill}
\begin{center}
% logo URL
\includegraphics[width=0.5\textwidth]{portada/\logoURL}~
\\[1cm]
\text{\LARGE Universidad Rafael Landívar}
\\
\text{\Large Facultad de Ingeniería}
\\
\text{\large Ingeniería en Informática y Sistemas}
\\[1cm]
\text{\large \nCurso} \\
\nAutor
\\[2cm]
\HBar \\[0.5cm]
\text{\LARGE\cTituloA}
\\[1cm]
\text{\LARGE\cTituloB} \\[0.5cm]
\HBar \\[3cm]

\cFecha
\end{center}
\vspace*{\fill}
\end{titlepage}
\pagenumbering{arabic}
